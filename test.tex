\documentclass{beamer}
%\documentclass[aspectratio=169]{beamer}  % inna proporcja slajdu

\usepackage[utf8]{inputenc}
\usepackage{polski}

\usetheme{pwrlite}
%\usetheme[nosections]{pwrlite}  % wyłączenie stron z sekcjami
%\usetheme[en]{pwrlite}  % logo w wersji angielskiej
%\usetheme[nosections,en]{pwrlite}  % logo w wersji angielskiej i bez sekcji

%\pdfmapfile{+lato.map}  % jeśli nie działa font Lato (a jest zainstalowany), odkomentuj przy pierwszej kompilacji

\title{Tytuł prezentacji}
\institute{Wydział \ldots\ Politechniki Wrocławskiej}
\author{Autor Prezentacji}
\date{data}

\begin{document}

\begin{frame}
\frametitle{Plan prezentacji}
\begin{itemize}
\item Pierwszy rozdział
\item Drugi rozdział\pause
\begin{itemize}
\item Pierwszy podrozdział
\item Drugi podrozdział
\end{itemize}
\end{itemize}
\end{frame}

\section{Pierwszy rozdział}

\begin{frame}
\frametitle{Wypunktowanie}
\begin{enumerate}
\item raz
\item dwa
\item trzy
\end{enumerate}
\end{frame}

\begin{frame}
\frametitle{Wzór}
$$\int_0^\infty \sin(x)\operatorname{d\!}{}x$$
\end{frame}

\end{document}

